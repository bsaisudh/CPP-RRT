\hypertarget{index_Overview}{}\section{Overview}\label{index_Overview}
Path planning for a point robot using Rapidly Exploring Random Trees (\hyperlink{classRRT}{R\+RT}) on a known 2D space. The algorithm returns coordinate points in the path, which when interfaced with a simple position control system can be used to drive a robot in the planned path. Path R\+R\+Ts are kinodynamic planners that can be used to calculate the trajectory of a robot in real time Given that the algorithm uses incremental motions, it can be used in Collision detection. The \hyperlink{classRRT}{R\+RT} algorithm can be used to produce good guesses for variational optimization techniques.\hypertarget{index_RRT}{}\section{A\+L\+G\+O\+R\+I\+T\+HM}\label{index_RRT}
1.\+Sample a random point from the configuration space. 2.\+Obtain a point on the tree closest to the sampled point, in the direction of the point at a unit distance. 3.\+Verify if this point is in contact with any obstacle. 4.\+If the new point is not in contact with or within any obstacles then add this point to the tree. 5.\+If the new point is in contact with or within any obstacles, then add the point closest to the tree that is just outside the obstacle to the tree. 6.\+The points added to the tree are removed from the sampling space. 7.\+This is recursively performed till the point within unit distance of the goal point is reached.\hypertarget{index_Project}{}\section{Specifics}\label{index_Project}
Programming Language -\/ C++ Build Platform – Make, G\+CC Compiler Source code control -\/ G\+IT and Git\+Hub Build testing – Travis CI Test coverage -\/ Coveralls\hypertarget{index_Agile}{}\section{Process}\label{index_Agile}
\href{https://docs.google.com/spreadsheets/d/1cJVLNv9pZ2T4a17OsMPn_WnxRS6tAkfYJKaMcSRo6MA/edit?usp=sharing}{\tt Sheet Link}\hypertarget{index_Class}{}\section{Diagram}\label{index_Class}
\hypertarget{index_License}{}\section{License}\label{index_License}
M\+IT License Copyright (c) 2018 Bala Murali Manoghar Sai Sudhakar Copyright (c) 2018 Akshay Rajaraman Permission is hereby granted, free of charge, to any person obtaining a copy of this software and associated documentation files (the \char`\"{}\+Software\char`\"{}), to deal in the Software without restriction, including without limitation the rights to use, copy, modify, merge, publish, distribute, sublicense, and/or sell copies of the Software, and to permit persons to whom the Software is furnished to do so, subject to the following conditions\+:

The above copyright notice and this permission notice shall be included in all copies or substantial portions of the Software.

T\+HE S\+O\+F\+T\+W\+A\+RE IS P\+R\+O\+V\+I\+D\+ED \char`\"{}\+A\+S I\+S\char`\"{}, W\+I\+T\+H\+O\+UT W\+A\+R\+R\+A\+N\+TY OF A\+NY K\+I\+ND, E\+X\+P\+R\+E\+SS OR I\+M\+P\+L\+I\+ED, I\+N\+C\+L\+U\+D\+I\+NG B\+UT N\+OT L\+I\+M\+I\+T\+ED TO T\+HE W\+A\+R\+R\+A\+N\+T\+I\+ES OF M\+E\+R\+C\+H\+A\+N\+T\+A\+B\+I\+L\+I\+TY, F\+I\+T\+N\+E\+SS F\+OR A P\+A\+R\+T\+I\+C\+U\+L\+AR P\+U\+R\+P\+O\+SE A\+ND N\+O\+N\+I\+N\+F\+R\+I\+N\+G\+E\+M\+E\+NT. IN NO E\+V\+E\+NT S\+H\+A\+LL T\+HE A\+U\+T\+H\+O\+RS OR C\+O\+P\+Y\+R\+I\+G\+HT H\+O\+L\+D\+E\+RS BE L\+I\+A\+B\+LE F\+OR A\+NY C\+L\+A\+IM, D\+A\+M\+A\+G\+ES OR O\+T\+H\+ER L\+I\+A\+B\+I\+L\+I\+TY, W\+H\+E\+T\+H\+ER IN AN A\+C\+T\+I\+ON OF C\+O\+N\+T\+R\+A\+CT, T\+O\+RT OR O\+T\+H\+E\+R\+W\+I\+SE, A\+R\+I\+S\+I\+NG F\+R\+OM, O\+UT OF OR IN C\+O\+N\+N\+E\+C\+T\+I\+ON W\+I\+TH T\+HE S\+O\+F\+T\+W\+A\+RE OR T\+HE U\+SE OR O\+T\+H\+ER D\+E\+A\+L\+I\+N\+GS IN T\+HE S\+O\+F\+T\+W\+A\+RE. 